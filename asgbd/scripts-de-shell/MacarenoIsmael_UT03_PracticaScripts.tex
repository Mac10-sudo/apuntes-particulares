% Created 2025-02-13 Thu 09:51
% Intended LaTeX compiler: pdflatex
\documentclass[11pt]{article}
\usepackage[utf8]{inputenc}
\usepackage[T1]{fontenc}
\usepackage{graphicx}
\usepackage{longtable}
\usepackage{wrapfig}
\usepackage{rotating}
\usepackage[normalem]{ulem}
\usepackage{amsmath}
\usepackage{amssymb}
\usepackage{capt-of}
\usepackage{hyperref}
\usepackage{minted}
\usepackage[margin=0.5in]{geometry}
\hypersetup{colorlinks, linkcolor=black}
\usepackage{geometry}
\geometry{top=2cm}
\geometry{bottom=2cm}
\usepackage{fancyhdr}
\pagestyle{fancy}
\fancyhf[L]{Administración de Sistemas Gestores de Bases de Datos}
\fancyhf[R]{Tema 01}
\fancyfoot[R]{ismael.macareno@educa.madrid.org}
\fancyfoot[L]{CC BY-NC-SA 4.0}
\usepackage{parskip}
\usepackage{mdframed}
\usepackage{fancyvrb}
\usepackage{xcolor}
\definecolor{shadecolor}{RGB}{220,220,220}
\newenvironment{shadedcode}{%
\VerbatimEnvironment
\begin{mdframed}[backgroundcolor=shadecolor,linewidth=0pt]}%
{\end{mdframed}}
\usepackage{attachfile2}
\newcommand{\textattachfilecolor}[2]{\textattachfile[color=0 0 0.5]{#1}{\textcolor{blue}{#2}}}
\usepackage[spanish]{babel}
\usepackage{datetime2}
\DTMlangsetup[es-ES]{ord=raise}
\renewcommand{\dateseparator}{/}
\usepackage{titlesec}
\usepackage{afterpage}
\newcommand\blankpage{\null\thispagestyle{empty}\newpage}
\usepackage{colortbl}
\usepackage{pdfpages}
\usepackage{tcolorbox}
\usepackage{listings}
\usepackage[spanish]{babel}
\lstset{
inputencoding=utf8,
extendedchars=true,
literate={ñ}{{\~n}}1 {Ñ}{{\~N}}1 {á}{{\'a}}1 {é}{{\'e}}1 {í}{{\'i}}1 {ó}{{\'o}}1 {ú}{{\'u}}1 {Á}{{\'A}}1 {É}{{\'E}}1 {Í}{{\'I}}1 {Ó}{{\'O}}1 {Ú}{{\'U}}1,
basicstyle=\ttfamily,
breaklines=true,
columns=fullflexible,
keepspaces=true,
language=TeX,
morekeywords={*, -, **, /}
}
\author{Ismael Macareno Chouikh}
\date{\today}
\title{Práctica \emph{scripts} \texttt{bash}}
\hypersetup{
 pdfauthor={Ismael Macareno Chouikh},
 pdftitle={Práctica \emph{scripts} \texttt{bash}},
 pdfkeywords={},
 pdfsubject={},
 pdfcreator={Emacs 29.4 (Org mode 9.6.15)}, 
 pdflang={Spanish}}
\begin{document}

\maketitle
\tableofcontents

\blankpage

\section{\emph{Scripts} de inicio y parada de Oracle}
\label{sec:orga4b3f8a}
\begin{minted}[linenos,firstnumber=1,frame=single]{bash}
#!/bin/bash

# Variables de entorno
export ORACLE_SID=asir
export ORACLE_HOME=/opt/oracle-install
export PATH=$ORACLE_HOME/bin:$PATH

# Iniciar el listener
lsnrctl start

# Arrancar oracle
sqlplus / as sysdba <<EOF
startup;
exit;
EOF
\end{minted}
\begin{minted}[linenos,firstnumber=1,frame=single]{bash}
#!/bin/bash
# variables de entorno
export ORACLE_SID=asir
export ORACLE_HOME=/opt/oracle-install
export PATH=$ORACLE_HOME/bin:$PATH

# Parar oracle
sqlplus / as sysdba <<EOF
shutdown inmediante;
exit;
EOF

# Parar el listener
lsnrctl stop
\end{minted}

\section{Arrancar automáticamente Oracle cuando se inicie el servidor}
\label{sec:org62cb794}
\begin{minted}[linenos,firstnumber=1,frame=single]{bash}
#!/bin/bash

# por si acaso, deshabilitamos SELinux para que no moleste
# modificar /etc/selinux/config para hacerlo permanente
setenforce 0

# ORACLE_HOME del sistema
valor_oracle_home=$(env | grep ORACLE_HOME | cut -d "=" -f 2)

# miro en oratab
valor_oracle_home_en_oratab=$(cat /etc/oratab | grep $valor_oracle_home | cut -d ":" -f 3)


# Habra que crear el fichero de logs si no existe
directorio_de_logs="/home/alumno/logs"
fichero_de_logs="/home/alumno/logs/oracle.log"
mkdir -p "$directorio_de_logs"


# tenia la fecha formateada con el echo normal, pero me apetece meter funciones
#    date "+%Y-%m-%d-%H:%M:%S" formato de fecha para "$fichero_de_logs"
fechaformateada (){
    date "+%Y-%m-%d-%H:%M:%S"
}


if [ -n "$valor_oracle_home_en_oratab" ] && [ $valor_oracle_home_en_oratab = "Y" ]; then
    echo "$(fechaformteada) Solicitud de arrancar Oracle" | tee -a "$fichero_de_logs"
    echo "$(fechaformteada) Oracle arrancado porque /etc/oratab indica Y" | tee -a "$fichero_de_logs"
    sqlplus / as sysdba <<EOF
startup;
EOF
    echo "$(fechaformteada) Oracle arrancado" | tee -a "$fichero_de_logs"
else
    echo "$(fechaformteada) Solicitud de arrancar Oracle" | tee -a "$fichero_de_logs"
    echo "$(fechaformteada) Oracle arrancado porque /etc/oratab indica N" | tee -a "$fichero_de_logs"
    echo "$(fechaformteada) Solicitud de parar Oracle" | tee -a "$fichero_de_logs"
    sqlplus / as sysdba <<EOF
shutdown inmediate;
EOF
    echo "$(fechaformteada) Oracle parado" | tee -a "$fichero_de_logs"    
fi
\end{minted}

\section{Crea usuarios de base de datos}
\label{sec:org6a73307}
\begin{minted}[linenos,firstnumber=1,frame=single]{bash}
#!/bin/bash

# select username, account_status from dba_users where lock_date is not null | para saber si un usuario esta bloqueado
# ALTER USER account ACCOUNT UNLOCK; | para desbloquear la cuenta

# Script para crear nuevos usuarios de oracle

# Variables de entorno
# Variables de entorno
export ORACLE_SID=zilae
export ORACLE_HOME=/opt/oracle-install
export PATH=$ORACLE_HOME/bin:$PATH


nombre_usuario=$1
contrasena_usuario=$2
pdb="pdzilae"

# comprobar si mete mas parametros de los que deberia
parametos_introducidos=$#

if [ "$parametos_introducidos" -ne 2 ]; then
    echo "Crea un usuario nuevo de oracle, con permisos connect y resource."
    echo "Si el usuario ya existe, lo desbloquea y le cambia la contraseña."
    echo ""
    echo "Uso: nuevo-usuario-oracle.sh <usuario> <contraseña>"
    sleep 10
    exit
fi


comprobar_si_existe_usuario_oracle (){
    local nombre_usuario="$1"

    resultado_query=$(sqlplus -s / as sysdba<<EOF
SET PAGESIZE 0 FEEDBACK OFF VERIFY OFF HEADING OFF ECHO OFF
SELECT COUNT(*) FROM dba_users WHERE USERNAME = UPPER('${nombre_usuario}');
EXIT;
EOF
                   )

    # recortar espacios en blanco
    echo "$resultado_query" | xargs
}

comprobar_si_usuario_bloqueado (){
    resultado_query_block=$(sqlplus -s / as sysdba<<EOF
SET PAGESIZE 0 FEEDBACK OFF VERIFY OFF HEADING OFF ECHO OFF
SELECT COUNT(*) FROM dba_users WHERE lock_date IS NOT NULL AND USERNAME = UPPER('${nombre_usuario}')
EXIT;
EOF
                         )

    echo "$resultado_query_block" | xargs
}

verificar_pdb(){
    estado_pdb=$(sqlplus -s / as sysdba<<EOF
SET PAGESIZE 0 FEEDBACK OFF VERIFY OFF HEADING OFF ECHO OFF
SELECT OPEN_MODE FROM v\$pdbs WHERE NAME = UPPER('${pdb}');
EXIT;
EOF
              )
    echo "$estado_pdb" | xargs
}

crear_usuario_dentro_de_oracle (){
    local estado_pdb=$(verificar_pdb)
    if [ "$estado_pdb" != "READ WRITE" ]; then
        sqlplus -s / as sysdba<<EOF
ALTER PLUGGABLE DATABASE ${pdb} OPEN;
EOF
    fi
    sqlplus -s / as sysdba<<EOF
ALTER SESSION SET CONTAINER=${pdb};
CREATE USER ${nombre_usuario} IDENTIFIED BY ${contrasena_usuario};
EXIT;
EOF
}

desbloquea_usu(){
sqlplus -s / as sysdba<<EOF
ALTER USER '${nombre_usuario}' ACCOUNT UNLOCK;
EXIT;
EOF
}

resultado_funcion_comprobar_usuario=$(comprobar_si_existe_usuario_oracle)


if [ "$resultado_funcion_comprobar_usuario" -gt 0 ]; then
    echo "El usuario $nombre_usuario ya existe"

    resultado_funcion_usuario_block=$(comprobar_si_usuario_bloqueado)
    if [ "$resultado_funcion_usuario_block" -eq 1 ]; then
        echo "El usuario $nombre_usuario esta bloqueado"
        echo "Desbloqueando......"
        desbloquea_usu
        echo "El usuario $nombre_usuario ha sido desbloqueado"
    fi
else
    echo "El usuario $nombre_usuario no existe"
    echo "Creando....."
    crear_usuario_dentro_de_oracle
    echo "El usuario $nombre_usuario ha sido creado"
fi
\end{minted}

\section{Almacena información periódicamente en la base de datos}
\label{sec:org9d9271a}
\begin{minted}[linenos,firstnumber=1,frame=single]{bash}
#!/bin/bash
# Almacena información periódicamente en la base de datos

# Variables de entorno para que chute sqlplus
export ORACLE_SID=zilae
export ORACLE_HOME=/opt/oracle-install
export PATH=$ORACLE_HOME/bin:$PATH

# Creacion de tablas
# Quiero comprobar si existe o no la tabla
crear_tablas(){
    # Si se quita el SET HEADING OFF da fallo diciendo que no existe la tabla DF
    tabla_existe=$(sqlplus -s / as sysdba<<EOF
SET HEADING OFF
SELECT COUNT(*) FROM all_tables WHERE table_name = 'DF';
EXIT;
EOF
                )
    if [ "$tabla_existe" -eq 0 ]; then
sqlplus -s / as sysdba<<EOF
create table DF(
hora varchar(40),
sistema varchar(40),
tamano varchar(40),
usado varchar(40),
montado varchar(40)
);
EXIT;
EOF

    else
        echo "Ya existe la tabla DF"
    fi
}

hora_actual=$(date '+%T')

crear_tablas

df -k | awk '{print $1, $2, $3, $6}' | tail -n +2 | while read -r sistema tamano usado montado; do
    insertar="INSERT INTO DF(hora, sistema, tamano, usado, montado) VALUES(
    '$hora_actual', '$sistema', '$tamano', '$usado', '$montado');"

    sqlplus -s / as sysdba<<EOF
$insertar
EOF
done

# df -k | awk '{print $1}' | tail -n +2
# df -k | awk '{print $2}' | tail -n +2
# df -k | awk '{print $3}' | tail -n +2
# df -k | awk '{print $6}' | tail -n +2
\end{minted}

\section{Envía un correo periódicamente}
\label{sec:org8be3f67}
\begin{minted}[linenos,firstnumber=1,frame=single]{bash}
#!/bin/bash
# Envía un correo periódicamente
# Enviar a alvaro@alvarogonzalez.no-ip.biz

# Variables de entorno
export ORACLE_SID=zilae
export ORACLE_HOME=/opt/oracle-install
export PATH=$ORACLE_HOME/bin:$PATH

sacar_informacion(){
    info=$(sqlplus -s / as sysdba<<EOF
set colsep ,     -- separate columns with a comma
set pagesize 0   -- No header rows
set trimspool on -- remove trailing blanks
set headsep off  -- this may or may not be useful...depends on your headings.
set linesize X   -- X should be the sum of the column widths
set numw X       -- X should be the length you want for numbers (avoid scientific notation on IDs)

spool myfile.csv
SELECT
        sistema, avg(tamano), avg(usado), montado
FROM
        DF
GROUP BY
      sistema, montado;
EOF
    )

# echo "$info" | xargs | sed 's/\s+,/,/' myfile.csv 
}

sacar_informacion

comprobacion_postfix(){
    comando=$(rpm -qa | grep postfix | wc -l)

    if [ "$comando" -ge 1 ]; then
        echo "POSTFIX esta instalado"
    else
        echo "POSTFIX NO esta instalado"
        echo "Instalando....."
        dnf install -y postfix mailx
        echo "Se ha instalado POSTFIX"
    fi
}

comprobacion_postfix

programar_tarea(){
    cron_schedule="* * * * * /home/oracle/Documents/practica-scripts/enviar-correo.sh"
}

programar_tarea

enviar_correo(){
fichero_csv="/home/oracle/Documents/practica-scripts/myfile.csv"
echo "Informe de la tabla DF" | mail -s "Ismael Macareno Chouikh" -a "$fichero_csv" alvaro@alvarogonzalez.no-ip.biz
}

enviar_correo
\end{minted}

\section{\emph{units} de \texttt{systemd}}
\label{sec:orgd3ff2db}
\textattachfile[color=0 0 1]{units.txt}{Adjunto fichero TXT} \newline

\section{Fichero \texttt{/etc/crontab}}
\label{sec:org6207534}
\textattachfile[color=0 0 1]{crontab.txt}{Adjunto fichero TXT} \newline
\end{document}
